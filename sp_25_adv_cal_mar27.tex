\documentclass{beamer}

\usepackage{amsmath, amssymb}
\usepackage{tikz-cd}
\usepackage{xcolor}
\usepackage{graphicx}

\title{MAT426: Advanced Calculus}
\author{\textbf{Miraj Samarakkody}}
\institute{Tougaloo College}
\date{03/27/2025}

\begin{document}

\begin{frame}
    \titlepage
\end{frame}



\begin{frame}{2.38 Theorem}
\begin{block}{Theorem}
If \(\{I_n\}\) is a sequence of intervals in \(\mathbb{R}^1\), such that \(I_n \supset I_{n+1}\) \((n=1,2,3,\dots)\), then \(\cap_{n=1}^\infty I_n\) is not empty. 
\end{block}
\begin{block}{Proof:}

\end{block}
\end{frame}

\begin{frame}{2.39 Theorem}
\begin{block}{Theorem}
Let \(k\) be a positive integer. If \(\{I_n\}\) is a sequence of \(k-\)cells such that \(I_n \supset I_{n+1}\) \((n=1,2,3,\dots)\), then \(\cap_{i=1}^\infty I_n\) is not empty. 
\end{block}
\begin{block}{Proof:}

\end{block}
\end{frame}

\begin{frame}{Theorem}
\begin{block}{Theorem}
Every \(k-\)cell is compact.
\end{block}
\begin{block}{Proof:}

\end{block}
\end{frame}

\begin{frame}
    \frametitle{2.41 Theorem}

    \begin{block}{Theorem}
        If a set \(E\) in \(\mathbb{R}^k\) has one of the following three properties, then it has the other two: 
        \begin{enumerate}[(a)]
            \item \(E\) is closed and bounded.
            \item \(E\) is compact. 
            \item Every infinite subset of \(E\) has a limit point in \(E\).
        \end{enumerate}
    \end{block}
    \begin{block}{Proof:}
        
    \end{block}

\end{frame}



\end{document}