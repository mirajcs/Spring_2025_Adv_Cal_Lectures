\documentclass{beamer}

\usepackage{amsmath, amssymb}
\usepackage{tikz-cd}
\usepackage{xcolor}
\usepackage{graphicx}

\title{MAT426: Advanced Calculus - Numerical Sequences and Series}
\subtitle{Convergent Sequences}
\author{\textbf{Miraj Samarakkody}}
\institute{Tougaloo College}
\date{Updated: \today}

\begin{document}

\begin{frame}
    \titlepage
\end{frame}





    \begin{frame}
        \frametitle{Convergent Sequences}
    
        \begin{block}{3.1 Definition}
            A sequence \(\{p_n\}\) in a metric space \(X\) is said to \textbf{converge} if there is a point \(p \in X\) with the following property: \\
            \vspace{0.2in}

            For every \(\epsilon>0\) there is an integer \(N\) such that \(n\geq N\) implies that \(d(p_n,p) < \epsilon\). 
        \end{block} \pause 
        \begin{itemize}
            \item We say that \(\{p_n\}\) converges to \(p\), or that \(p\) is the limit of \(p_n\). \pause
            \item We write \(p_n \to p \) or \(\lim_{n \to \infty}p_n =p\). \pause
            \item If \(\{p_n\}\) does not converge, it is said to \textit{diverge.}\pause
            \item Convergent sequence depends not only on \(\{p_n\}\) but also on \(X\)\\
            Ex: Sequence \(1/n\) in \(\mathbb{R}^1\) and in set of all positive real numbers with \(d(x,y)=|x-y|\)
        \end{itemize}
    \end{frame}

\begin{frame}
    \frametitle{Convergent Sequences}

    \begin{itemize}
        \item The set of all points \(p_n~(n=1,2,3, \dots)\) is the range of \(\{p_n\}. \)
        \item The range of a sequence may be finite set, or may be infinite. 
        \item The sequence \(\{p_n\}\) is said to be bounded if its range is bounded. 
    \end{itemize}

\end{frame}

\begin{frame}
    \frametitle{Example}

    Consider the following sequences of complex numbers (\(X = \mathbb{R}^2\)): \pause
    \begin{itemize}
        \item If \(s_n=1/n\), then \(\lim_n \to \infty s_n =0\); the range is infinite and the sequence is bounded. \pause
        \item If \(s_n=n^2\), the sequence \(\{s_n\}\) is unbounded, is divergent, and has infinite range. \pause
        \item If \(s_n= 1+ [(-1)^n/n]\), the sequence \(\{s_n\}\) converges to \(1\), is bounded, and has infinite range. \pause  
        \item If \(s_n = i^2\), the sequence \(\{s_n\}\) is divergent, is bounded, and has finite range. \pause
        \item If \(s_n = 1~(n=1,2,3, \dots)\), then \(\{s_n\}\) converges to 1, is bounded, and has finite range. \pause
        \item If \(s_n =1 ~(n=1,2,3, \dots)\), then \(\{s_n\}\) converges 1, is bounded, and has finite range. 
    \end{itemize}

\end{frame}

\begin{frame}
    \frametitle{Theorem}

    Let \(\{p_n\}\) be a sequence in a metric space \(X\). 

    \begin{itemize}
        \item \(\{p_n\}\) converges to \(p \in X\) if and only if every neighborhood of \(p\) contains \(p_n\) for all but finitely many \(n\). \pause
        \item If \(p \in X\), \(p' \in X\), and if \(\{p_n\}\) converges to \(p\) and to \(p'\), then \(p'=p\).\pause
        \item If \(\{p_n\}\) converges, then \(\{p_n\}\) is bounded. \pause
        \item If \(E \subset X \) and if \(p\) is a limit point of \(E\), then there is a sequence \(\{p_n\}\) in \(E\) such that \(p = \lim_{n \to \infty}p_n\). 
    \end{itemize}

\end{frame}

\begin{frame}
    \frametitle{3.3 Theorem}

    Suppose \(\{s_n\},\{t_n\}\) are complex sequences, and \(\lim_{n\to \infty}s_n =s\), \(
    \lim_{n \to \infty}t_n =t\). Then: \pause 
    \begin{itemize}
        \item \(\lim_{n \to \infty} (s_n + t_n)=s +t\) \pause
        \item \(\lim_{n \to \infty} cs_n= cs\) for any number \(c\). \pause
        \item \(\lim_{n \to \infty} (c\pm s_n)= c \pm s\) for any number \(c\). \pause
        \item \(\lim_{n \to \infty} (s_n t_n) = st\) \pause
        \item \(\lim_{n \to \infty} (1/t_n) = 1/t\) provided \(t_n \ne 0\) and \(t \neq 0\).
    \end{itemize}

\end{frame}

\begin{frame}
    \frametitle{3.4 Theorem}

    \begin{itemize}
        \item Suppose \(\mathbf{x}_n \in \mathbb{R}^k~(n=1,2,3, \dots)\) and \[\mathbf{x}_n = (\alpha_{1,n}, \dots, \alpha_{k,n}).\] Then \(\{\mathbf{x}_n\}\) converges to \(\mathbf{x} = (\alpha_1, \dots, \alpha_k)\) if and only if \(\lim_{n \to \infty}\alpha_{i,n} = \alpha_i\) for \(i=1,2,\dots,k\). \pause 
    \end{itemize}

\end{frame}


\end{document}
