\documentclass{beamer}

\usepackage{amsmath, amssymb}
\usepackage{tikz-cd}
\usepackage{xcolor}
\usepackage{graphicx}

\title{MAT426: Advanced Calculus}
\author{\textbf{Miraj Samarakkody}}
\institute{Tougaloo College}
\date{04/01/2025}

\begin{document}

\begin{frame}
    \titlepage
\end{frame}





\begin{frame}
    \frametitle{2.41 Theorem}

    \begin{block}{Theorem}
        If a set \(E\) in \(\mathbb{R}^k\) has one of the following three properties, then it has the other two: 
        \begin{enumerate}[(a)]
            \item \(E\) is closed and bounded.
            \item \(E\) is compact. 
            \item Every infinite subset of \(E\) has a limit point in \(E\).
        \end{enumerate}
    \end{block}
    \begin{block}{Proof:}
        
    \end{block}

\end{frame}

\begin{frame}
    \frametitle{2.42 Theorem}

    \begin{block}{Theorem - Weierstrass}
        Every bounded infinite subset of \(\mathbb{R}^k\) has a limit point in \(\mathbb{R}^k\).
    \end{block}

\end{frame}

\begin{frame}
    \frametitle{}

        \Huge{Perfect Sets}

\end{frame}

\begin{frame}
    \frametitle{2.43 Theorem}

    Let \(P\) be a non-empty perfect set in \(\mathbb{R}^k\). Then \(P\) is uncountable. 

\end{frame}

\begin{frame}
    \frametitle{Corollary}

    \begin{block}{Corollary}
        Every interval \([a,b]\) is uncountable. In particluar, the set of all real numbers is uncountable. 
    \end{block}

\end{frame}

\begin{frame}
    \frametitle{The Cantor Set}

    Cantor set is a perfect set in \(\mathbb{R}^1\) which contain no segment. \\ \pause 
    \vspace{0.2in}
    Let \(E_0\) be the interval \([0,1]\). \\ \pause
    \vspace{0.2in}
    Remove the segment \(\left(\dfrac{1}{3}, \dfrac{2}{3}\right)\), and let \(E_1\) be the union of the intervals \[\left[0, \dfrac{1}{3}\right], \left[\dfrac{2}{3}\right]\]\pause
    Remove the middle thirds of these intervals, and let \(E_2\) be the union of the intervals \[\left[0, \dfrac{1}{9}\right], \left[\dfrac{2}{9}, \dfrac{3}{9}\right],\left[\dfrac{6}{9}, \dfrac{7}{9}\right],\left[\dfrac{8}{9}, 1\right]\]


\end{frame}

\begin{frame}
    \frametitle{The Cantor set}

    Continuing in this way, we obtain a sequence of compact sets \(E_n\), such that \begin{enumerate}
        \item \(E_1 \supset E_2 \supset E_3 \supset \dots\)
        \item \(E_n\) is the union of \(2^n\) intervals, each of length \(3^{-n}\).
    \end{enumerate}
The set \[P = \cap_{n=1}^\infty E_n\] is called the \textbf{Cantor set}.
\end{frame}


\end{document}