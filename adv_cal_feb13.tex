\documentclass{beamer}

\usepackage{amsmath, amssymb}
\usepackage{tikz-cd}
\usepackage{xcolor}
\usepackage{graphicx}

\title{MAT426: Advanced Calculus}
\author{\textbf{Miraj Samarakkody}}
\institute{Tougaloo College}
\date{02/13/2025}

\begin{document}

\begin{frame}
    \titlepage
\end{frame}

\begin{frame}{Finite, Countable and Uncountable Sets}
    \begin{block}{Definition}
        For any positive integer \(n\), let \(J_n\) be the set whose elements are the integers \(1, 2, \ldots, n\). Let \(J\) be the set consisting of all positive integers. For any set \(A\) we say, 
        \begin{enumerate}
            \item \(A\) is finite \(A \sim J_n\) for some \(n\). (the empty set is also considered to be finite)
            \item \(A\) is infinite if \(A\) is not finite.
            \item \(A\) is countable if \(A \sim J\). 
            \item \(A\) is uncountable if \(A\) is neither finite nor countable.
            \item \(A\) is atmost countable if \(A\) is either finite or countable.
        \end{enumerate}
    \end{block}
    A set \(A\) is said to be \textbf{finite} if it is empty or if there is a one-to-one correspondence between \(A\) and \(J_n\) for some positive integer \(n\). A set that is not finite is said to be \textbf{infinite}.

\end{frame}

\begin{frame}
    \frametitle{Example}
    Let \(A\) be the set of integers. Then \(A\) is countable. Consider the following arrengement of the sets \(A\) and \(J\). 
    \[
    A:0,1,-1,2,-2,3,-3,\dots
    \]
    \[J: 1,2,3,4, \dots\]
    Find a explicit formula for a function \(f\) from \(J\) to \(A\).\\ \pause
    \vspace{0.1in}
    Importance of this example
\end{frame}

\begin{frame}
    \frametitle{}
\begin{block}{Definition}
    By a sequence we mean a function \(f\) defined on the set \(J\) of all positive integers. If \(f(n) = a_n\) for all \(n\), we write \(\{a_n\}\) for the sequence \(\{a_1, a_2, a_3, \ldots\}\). The number \(a_n\) is called the \(n\)th term of the sequence.
\end{block}
    

\end{frame}


\end{document}